\section{Experiment}
\label{sec:experiment}
With all infrastructure and toolchain prepared, we perform a series of experiment
to demonstrate that ConAir is able to fix concurrency bugs. We will first
illustrate the experimnt results from the author of ConAir. After that we will
show our own experiment results for non-deadlock, deadlock programs and programs
with other concurrenty bugs.

\subsection{Original result}
Table~\ref{tab:original} lists the experiment result from the author. In the
ConAir paper the author evaluated ConAir on 10 real life programs, whose lines
of code varies from 1.2K to 693K. All these programs have known failures to
ConAir. The author run ConAir on these programs in two modes. The fix mode is
used when the programmer know where the bug is, and manually insert some
assertions to help ConAir identify and fix the bug; The survival mode is to run
ConAir without any information of the program, and let ConAir find and fix all
potential bugs on its own.

% Table generated by Excel2LaTeX from sheet 'Sheet1'
\begin{table}[htbp]
  \centering
  \caption{Add caption}
    \begin{tabular}{rrrrr}
    \toprule
    App. & LOC & Failures & Fix & Survival \\
    \midrule
    FFT & 1.2K & w.output & Yes & Conditional \\
    HawlNL & 10K & Hang & Yes & Yes \\
    HTTrack & 55K & Seg.fault & Yes & Yes \\
    MozillaXP & 112K & Seg.fault & Yes & Yes \\
    MozillaJS & 120K & Hang & Yes & Yes \\
    MYSQL1 & 682K & w.output & Yes & Conditional \\
    MYSQL2 & 693K & assertion & Yes & Yes \\
    SQLite & 95K & assertion & Yes & Yes \\
    Transmission & 67K & hang & Yes & Yes \\
    ZSNES & 37K & assertion & Yes & Yes \\
    \bottomrule
    \end{tabular}%
  \label{tab:addlabel}%
\end{table}%


From Table~\ref{tab:original} we can observe that ConAir is able to fix all
concurrency bugs for all these 10 real life programs in fix mode. However, for
wrong output failure, ConAir can only conditionally fix the bug in survival
mode, which means that ConAir needs additinal assertion information to know what
the correct output should be, otherwise ConAir cannot fix the bug. This
experiment shows that ConAir is capable of recover and fix concurrency bugs of
all know types, regardless of program sizes.

\subsection{Non-deadlock programs}
In order to further demonstrate how ConAir works for fixing different bugs, we
come up with our own concurrency programs with different types of bugs, and run
ConAir to see if it can fix the bug or not. Table~\ref{tab:nondl} lists all
programs with their corresponding bugs, and the result of running the original
executable and fixed executable. Since concurrency bugs do not always happen,
for each program we let the program run for 1000 types in fix mode, which means
  we add some assertions for the wrong output bug, and we record the number of
  correct outputs.

% Table generated by Excel2LaTeX from sheet 'Sheet2'
\begin{table}[htbp]
  \centering
    \begin{tabular}{|r|r|r|r|} \hline
    Program & Failures & Before fix & After fix \\ \hline
    WAW1 & w.output & 31/1000 & 1000/1000 \\
    WAW2 & w.output & 983/1000 & 1000/1000 \\
    RAW1 & Seg fault & 938/1000 & 1000/1000 \\
    RAW2 & Seg fault & 998/1000 & 1000/1000 \\
    RAR1 & Seg fault & 1000/1000 & 1000/1000 \\
    RAR2 & Seg fault & 1000/1000 & 1000/1000 \\
    WAR1 & assertion & 16/1000 & 1000/1000 \\
    WAR2 & assertion & 976/1000 & 1000/1000 \\ \hline
    \end{tabular}%
  \caption{Experiment result for non-deadlock programs for 1000 runs each}
  \label{tab:nondl}%
\end{table}%


From Table~\ref{tab:nondl} we can see that ConAir is capable to fix all three
known type of non-deadlock concurrency bugs. For example, for WAR1 only 31 out
of 1000 runs give correct output for the original executable. However, all 1000
runs give correct output after fix. We can observe similar behaviors for other
programs and other bugs. One thing to note is for RAR1 and RAR2 even for the
original program we cannot observe any error in 1000 runs, although we are sure
thretically segmentation faults will happen, therefore it is not a counter
exapml, therefore it is not a counter exapmle.

\subsection{Deadlock programs}
Different from non-deadlock programs, deadlock programs will usually hang up
instead of hault immediately and give the wrong output. Therefore we design
another experiment to demonstrate the ability of ConAir to fix deadlock bugs.
We come up with two simple programs with deadlock. A code sinppet of Deadlock1
is shown in Figure~\ref{fig:deadlock-code}.
\begin{figure}[htbp]
\begin{center}
\begin{lstlisting}[language=C]
              void thread1(void){
                pthread_mutex_lock(&lock1);
                pthread_mutex_lock(&lock2);
                pthread_mutex_unlock(&lock1);
                pthread_mutex_unlock(&lock2);
                printf("t1 done\n");
              }
              void thread2(void){
                pthread_mutex_lock(&lock2);
                pthread_mutex_lock(&lock1);
                pthread_mutex_unlock(&lock1);
                pthread_mutex_unlock(&lock2);
                printf("t2 done\n");
              }
\end{lstlisting}
\caption{Code snippet of a deadlock bug}
\end{center}
\label{fig:deadlock-code}
\end{figure}

We run ConAir on the two programs with deadlock for 1000 times, and we will
assume no deadlock happens if it does not hang for all 1000 times. The results
are shown in Table~\ref{tab:dl}.

% Table generated by Excel2LaTeX from sheet 'Sheet3'
\begin{table}[htbp]
  \centering
    \begin{tabular}{|r|r|r|} \hline
    Program & Before fix & After fix \\ \hline
    Deadlock1 & Hang & Hang \\ \hline
    Deadlock2 & Hang & Hang \\ \hline
    \end{tabular}%
  \caption{Experiment result for deadlock programs}
  \label{tab:dl}%
\end{table}%


From Table~\ref{tab:dl} we can see that both the initial executable and fixed
executable has deadlock programs. We tried to look into the ConAir code as well
as its configuration file, and we believe all of our settings are correct, and
therefore it might be that ConAir cannot fix the deadlock program that we have.

We contacted the author with this question, together with our command,
configuration file and our buggy programs. They told us that for running ConAir
for deadlock programs some other commands need to be added, and they will
  provide me some further explaination later. Therefore, we still believe that
  ConAir is capable to fix some deadlock bugs, especially the one that we have. However, in the scope of this project we cannot demonstrate it yet.

Also, in the ConAir paper the author claims that ConAir is not able to fix all
deadlock bugs, if the deadlock cannot be fixed by reexecute the idempotent
region. This claim make us to think that there may be other concurrenty bugs
that ConAir cannot fix. We design another set of experiment in
Section~\ref{sec:ex-other}, where we show how ConAir performs for other unknown
types of concurrenty bugs.

\subsection{Programs with other concurrenty bugs}
\label{sec:ex-other}
% Table generated by Excel2LaTeX from sheet 'Sheet4'
\begin{table*}[htbp]
  \centering
\begin{tabular}{|c|p{7cm}|r|r|} \hline
  Program & Failures & Before fix & After fix \\ \hline
  bug1 & More than one thread is waiting for a Cond signal, but only one is sent & Hang & Hang \\ \hline
  bug2 & It has exhausted the default thread stack space & Seg.fault & Seg.fault \\ \hline
  bug3 & Shows an unsafe way to pass arguments in the pthread\_create, pass a memory location which value is not unique & w.output & Try to fix but failed \\ \hline
  bug4 & Synchronization problem,  signal thread calling before waiting & Hang & Fix \\ \hline
  bug5 & Missing a pthread\_exit call at the end of main & w.output & w.output \\ \hline
  bug6 & Read and write critical section without synchronization & w.output (random) & w.output (stable) \\ \hline
\end{tabular}%
\caption{Experiment results for other concurrency bugs}
\label{tab:other}%
\end{table*}%

In the above experiments we show that ConAir is able to fix all known
non-deadlock bugs, and we still believe it can fix the deadlock bugs if we use
the right command. However, we would also want to understand how ConAir deals
with other types of concurrenty bugs.

However, it is not easy to find a good benchmark that satisfies our requirement.
Most of the state-of-art concurrency benchmarks are designed to test the
performance of CPU simulators, and they are all bug free. We do some literateral
search but cannot find any simple set of buggy multithreaded programs.
Fortunately, we are able to find a pthread programming
exercise~\cite{pthread-web}, where it provides six buggy multithreaded programs
with very different types of bugs from the ones that are supported by ConAir.
Table~\ref{tab:other} lists all of the six programs and the description of each
corresponding bug, as well as the execution result before and after running
ConAir on the buggy program.

From Table~\ref{tab:other} we can see that ConAir cannnot fix most of the
unknown bugs. For bug1, bug2 and bug5 the results before and after fix are
exactly the same. However, we could observe that although some bug types are
unknown, ConAir will try to fix them. For example, for bug3 we can observe from
the result that ConAir is trying to insert the setjmp() and longjmp() function,
but just cannot fix that. For bug6 ConAir is providing a more stable output,
compared with a pure random output before fix. For bug 4 ConAir can even fix the
bug without knowing it can supporting that type of bug.
