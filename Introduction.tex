\section{Introduction}

\subsection{Background}
Multi-thread programming becomes more and more popular with the advent of
multi-core. With multi-core, multi-thread programming has a large advantage over
its CPU utility and efficiency. From the low level operating systen (OS)to high
level application, multi-thread gains its popularity. A lot of elements should
be considered in multi-thread programming, such as threads synchronization, the
protection of critical section and the prevention of deadlock. Whereas, the
debug of multi-thread programming is very hard because some bugs may happen in
certain execution of threads. Unfortunately, the order of threads exectution is
decided by scheduler of OS, which is limitedly revised by users. So it's very
hard for a user to claim he or she writes a bug-free multi-thread program. A
very common way to debug it is to execute the program for many times, such as
tens of, hundreds of or even thousands of, and then compare the output. But the
problem is still there. If one program is executed for ten times with the same
output, there's still some possibility than it will crush on 11th try. Most
multi-thread programming program is hidden in it. Even though, the user knows
there are some bugs in his or her programs, it's still very difficult to solve
it due to its uncertain execution order. One bug may be caused by one of thread
or by multiple ones. In that case, with different execution order, user can not
decide where the bug is or how to solve it.  In general, multi-thread
programming is very commonly used nowadays and the debug of it is very
troublesome and time-consuming. The design of ConAir is based on users'
requirements. It is an auto concurrency bug recovery tool which can be used to
recovery a buggy multi-thread program.

\subsection{Standards}
There are four standards which are used to judge a tool's effectiveness, including:\\
\textbf{Compatibility}: It is used to judge whether there is OS/Hardware modification for the tool. Some tools need low level modification, for example, the change of scheduler of OS. Wheras, OS is designed to be stable and limitedly changeable. If a tool needs OS modification, it may cause some future problems and demage the stableness of the whole system. Sometimes, it may even cause the crush of the system.\\
\textbf{Correctness}: It is used to determine whether the usage of a tool will generate results infeasible for original software. If so, the usage of a tool causes more problems.\\
\textbf{Generality}: With high generality, it indicates the tool can solve a wide variety of problems. The effectiveness of a tool is judged based on it because for multi-thread programming, problems are varied. If a tool can solve most common types of concurrency bugs, it is more feasible and practical.\\
\textbf{Performance}: It is used to tell whether the tool adds small amount of overhead to the original program and has an ability of fast failure recovery. \\
The design of ConAir takes all four standards into consideration. It does not require the change of OS or hardware. It guarantees not adding more bugs in origin software. It can solve most types of concurrency bugs and it limits the overhead to the least amount of time. 
