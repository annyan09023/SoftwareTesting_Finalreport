\section{Discussions and Future Work}
In this project we explored the state-of-art research tool ConAir, which is used
for concurrency bug recovery. We believe that there are many dimensions that the
  current work for both ConAir and our project can be extended.

\subsection{Better compatibility}
In this project we have a very painful experience setting up ConAir and make it
running. One of the reasons is that ConAir is built based on old version of
Linux and LLVM, and it does not recognize the latest compilers and libraries.
Since ConAir is a very effective tool, it is worthy to spend time to maintain
this tool so that it is compatible for modern OS/software, and therefore it
could be more widely used.

\subsection{New type of concurrency bugs}
From our experiment result in Section\ref{sec:ex-other} we can see that for some
other types of concurrency bugs ConAir is still capable in fixing some of them.
It is promising that ConAir being extended to support new types of concurrency
bugs. For example, bug4 in previous experiment has a synchronization problem,
which is not any of the four known concurrency bugs. However, ConAir is able to
fix that bug. In the future we can identify more such bugs and extend ConAir to
add support to those bugs.

\subsection{Concurrent program generator}
In this project we generate all our buggy concurrent programs by hand. However,
during this process we feel that many of these programs are very similar in
format. It motivates us to have an idea of building a program generator for
concurrent programs for both bug-free and buggy programs. Specifically, we could
provide some parameters, such as number of threads, the work to be performed for
each thread, different thread scheduling, etc. If this current program generator
can be built, it could be used to evaluate many concurrent program testing and
bug recovery tool such as ConAir, and it will be of grate value for many
researches.
