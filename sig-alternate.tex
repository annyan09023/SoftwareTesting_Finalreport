% This is "sig-alternate.tex" V2.0 May 2012
% This file should be compiled with V2.5 of "sig-alternate.cls" May 2012
%
% This example file demonstrates the use of the 'sig-alternate.cls'
% V2.5 LaTeX2e document class file. It is for those submitting
% articles to ACM Conference Proceedings WHO DO NOT WISH TO
% STRICTLY ADHERE TO THE SIGS (PUBS-BOARD-ENDORSED) STYLE.
% The 'sig-alternate.cls' file will produce a similar-looking,
% albeit, 'tighter' paper resulting in, invariably, fewer pages.
%
% ----------------------------------------------------------------------------------------------------------------
% This .tex file (and associated .cls V2.5) produces:
%       1) The Permission Statement
%       2) The Conference (location) Info information
%       3) The Copyright Line with ACM data
%       4) NO page numbers
%
% as against the acm_proc_article-sp.cls file which
% DOES NOT produce 1) thru' 3) above.
%
% Using 'sig-alternate.cls' you have control, however, from within
% the source .tex file, over both the CopyrightYear
% (defaulted to 200X) and the ACM Copyright Data
% (defaulted to X-XXXXX-XX-X/XX/XX).
% e.g.
% \CopyrightYear{2007} will cause 2007 to appear in the copyright line.
% \crdata{0-12345-67-8/90/12} will cause 0-12345-67-8/90/12 to appear in the copyright line.
%
% ---------------------------------------------------------------------------------------------------------------
% This .tex source is an example which *does* use
% the .bib file (from which the .bbl file % is produced).
% REMEMBER HOWEVER: After having produced the .bbl file,
% and prior to final submission, you *NEED* to 'insert'
% your .bbl file into your source .tex file so as to provide
% ONE 'self-contained' source file.
%
% ================= IF YOU HAVE QUESTIONS =======================
% Questions regarding the SIGS styles, SIGS policies and
% procedures, Conferences etc. should be sent to
% Adrienne Griscti (griscti@acm.org)
%
% Technical questions _only_ to
% Gerald Murray (murray@hq.acm.org)
% ===============================================================
%
% For tracking purposes - this is V2.0 - May 2012

\documentclass{sig-alternate}

\pdfpagewidth=8.5in
\pdfpageheight=11in

\usepackage{lmodern}
\usepackage{amsmath}
\usepackage{booktabs, multicol, multirow}
\usepackage{verbatim}
\usepackage{enumerate}
\usepackage{hyperref}
\usepackage{subfigure}
\usepackage{color}
\usepackage[linesnumbered,ruled,vlined]{algorithm2e}
\usepackage{url}
\usepackage{listings}
\usepackage[T1]{fontenc}
\usepackage{xcolor}
\lstset{escapeinside={<@}{@>}}
\usepackage{rotating}
\usepackage{cite}
\usepackage{caption}


\definecolor{dkgreen}{rgb}{0,0.6,0}
\definecolor{gray}{rgb}{0.5,0.5,0.5}
\definecolor{mauve}{rgb}{0.58,0,0.82}
\definecolor{light-gray}{gray}{0.8}

\lstset{
  frame=none,
  language=C,
  aboveskip=3mm,
  belowskip=3mm,
  showstringspaces=false,
  columns=flexible,
  basicstyle={\small\ttfamily},
  %numbers=left,
  %numberstyle=\tiny\color{gray},
  keywordstyle=\color{blue},
  commentstyle=\color{dkgreen},
  stringstyle=\color{mauve},
  breaklines=true,
  breakatwhitespace=true
  tabsize=3
}

\begin{document}
%
% --- Author Metadata here ---
%\conferenceinfo{ASE}{'97 El Paso, Texas USA}
%\CopyrightYear{2007} % Allows default copyright year (20XX) to be over-ridden - IF NEED BE.
%\crdata{0-12345-67-8/90/01}  % Allows default copyright data (0-89791-88-6/97/05) to be over-ridden - IF NEED BE.
% --- End of Author Metadata ---

\title{ConAir: A Lightweight Concurrenty Bug Recovery Tool}
%\subtitle{[Extended Abstract]
%\titlenote{A full version of this paper is available as
%\textit{Author's Guide to Preparing ACM SIG Proceedings Using
%\LaTeX$2_\epsilon$\ and BibTeX} at
%\texttt{www.acm.org/eaddress.htm}}}
%
% You need the command \numberofauthors to handle the 'placement
% and alignment' of the authors beneath the title.
%
% For aesthetic reasons, we recommend 'three authors at a time'
% i.e. three 'name/affiliation blocks' be placed beneath the title.
%
% NOTE: You are NOT restricted in how many 'rows' of
% "name/affiliations" may appear. We just ask that you restrict
% the number of 'columns' to three.
%
% Because of the available 'opening page real-estate'
% we ask you to refrain from putting more than six authors
% (two rows with three columns) beneath the article title.
% More than six makes the first-page appear very cluttered indeed.
%
% Use the \alignauthor commands to handle the names
% and affiliations for an 'aesthetic maximum' of six authors.
% Add names, affiliations, addresses for
% the seventh etc. author(s) as the argument for the
% \additionalauthors command.
% These 'additional authors' will be output/set for you
% without further effort on your part as the last section in
% the body of your article BEFORE References or any Appendices.

\numberofauthors{2} %  in this sample file, there are a *total*
% of EIGHT authors. SIX appear on the 'first-page' (for formatting
% reasons) and the remaining two appear in the \additionalauthors section.
%
\author{
  % You can go ahead and credit any number of authors here,
  % e.g. one 'row of three' or two rows (consisting of one row of three
  % and a second row of one, two or three).
  %
  % The command \alignauthor (no curly braces needed) should
  % precede each author name, affiliation/snail-mail address and
  % e-mail address. Additionally, tag each line of
  % affiliation/address with \affaddr, and tag the
  % e-mail address with \email.
  %
  % 1st. author
  \alignauthor Shiyu Dong\\
  \affaddr{The University of Texas at Austin}\\
  \email{shiyud@utexas.edu}\\
  % 2nd. author
  \alignauthor Xuebin Yan\\
  \affaddr{The University of Texas at Austin}\\
  \email{annyan@utexas.edu}\\
}
% There's nothing stopping you putting the seventh, eighth, etc.
% author on the opening page (as the 'third row') but we ask,
% for aesthetic reasons that you place these 'additional authors'
% in the \additional authors block, viz.
% Just remember to make sure that the TOTAL number of authors
% is the number that will appear on the first page PLUS the
% number that will appear in the \additionalauthors section.

\maketitle
\begin{abstract}
Concurrency bugs are usually hidden in software and is not really easy to be
exposed. These bugs often causes severe failure to end-users but they are
usually hard to fix by developers. ConAir~\cite{zhang2013conair} is a tool
targeting on recover and fix concurrent bugs for end users. It is based on two
observations that rolling back a single thread by re-executing idempotent region
for that thread is enough for recovering many of the concurrent bugs.

In this project we build and run ConAir from source code provided by the author.
Experiments from the authors of ConAir shows that ConAir is capable to fix and
recover concurrency bugs in many real life programs with different sizes, and
our own experiment results demonstrate that other than recovering the known four
types of concurrency bugs, ConAir has potential to fix more unknown types of
bugs.
\end{abstract}


%A category including the fourth, optional field follows...
\category{D.1.3}{Programming Techniques}{Concurrent Programming}
% A category with the (minimum) three required fields
\category{D.2.5}{Software Engineering}{Testing and Debugging}
\category{D.4.1}{Operating Systems}{Process Management}


\keywords{idempotency, concurrency bugs, failure recovery, static analysis, bug
fixing}

\section{Introduction}

\subsection{Background}
Multi-thread programming becomes more and more popular with the advent of
multi-core. With multi-core, multi-thread programming has a large advantage over
its CPU utility and efficiency. From the low level operating systen (OS)to high
level application, multi-thread gains its popularity. A lot of elements should
be considered in multi-thread programming, such as threads synchronization, the
protection of critical section and the prevention of deadlock. Whereas, the
debug of multi-thread programming is very hard because some bugs may happen in
certain execution of threads. Unfortunately, the order of threads exectution is
decided by scheduler of OS, which is limitedly revised by users. So it's very
hard for a user to claim he or she writes a bug-free multi-thread program. A
very common way to debug it is to execute the program for many times, such as
tens of, hundreds of or even thousands of, and then compare the output. But the
problem is still there. If one program is executed for ten times with the same
output, there's still some possibility than it will crush on 11th try. Most
multi-thread programming program is hidden in it. Even though, the user knows
there are some bugs in his or her programs, it's still very difficult to solve
it due to its uncertain execution order. One bug may be caused by one of thread
or by multiple ones. In that case, with different execution order, user can not
decide where the bug is or how to solve it.  In general, multi-thread
programming is very commonly used nowadays and the debug of it is very
troublesome and time-consuming. The design of ConAir is based on users'
requirements. It is an auto concurrency bug recovery tool which can be used to
recovery a buggy multi-thread program.

\subsection{Standards}
There are four standards which are used to judge a tool's effectiveness, including:\\
\textbf{Compatibility}: It is used to judge whether there is OS/Hardware modification for the tool. Some tools need low level modification, for example, the change of scheduler of OS. Wheras, OS is designed to be stable and limitedly changeable. If a tool needs OS modification, it may cause some future problems and demage the stableness of the whole system. Sometimes, it may even cause the crush of the system.\\
\textbf{Correctness}: It is used to determine whether the usage of a tool will generate results infeasible for original software. If so, the usage of a tool causes more problems.\\
\textbf{Generality}: With high generality, it indicates the tool can solve a wide variety of problems. The effectiveness of a tool is judged based on it because for multi-thread programming, problems are varied. If a tool can solve most common types of concurrency bugs, it is more feasible and practical.\\
\textbf{Performance}: It is used to tell whether the tool adds small amount of overhead to the original program and has an ability of fast failure recovery. \\
The design of ConAir takes all four standards into consideration. It does not require the change of OS or hardware. It guarantees not adding more bugs in origin software. It can solve most types of concurrency bugs and it limits the overhead to the least amount of time. 

\section{ConAir Overview}
\label{chp:Overview}
\subsection {Two observations}
The design of ConAir is based on two observations:
\begin{itemize}
\item
Roll back a single thread is sufficient to recover from most concurrency-bug failures.
\item
Re-execute an idempotent region is sufficient to recover from many concurrency-bug failures.
\end{itemize}
\subsection{Observation I}
For most concurrency bugs, re-executing one failing thread is sufficient to fix bugs.In the following part, most common concurrency bugs will be discussed separately:\\
\subsubsection{Recovering atomicity-violation bugs}
Atomicity violations contribute to about 70\% of real-word non-deadlock
bugs~\cite{lu2008learning}. For read and write operations, there are four kinds of atomic violation operations, including Read after Write (RAW), Read after Read (RAR), Write after Read (RAW) and Write after Write (WAW).

\begin{figure}
  \begin{center}
    \begin{tabular}{cc}
      \subfigure[\label{fig:atomic:RAW}RAW atomic violation]{\includegraphics[width=0.25\textwidth]{figs/RAW.png}}&
      \hspace{-7mm}\subfigure[\label{fig:atomic:WAW}WAW atomic violation]{\includegraphics[width=0.25\textwidth]{figs/WAW.png}} \\
      \subfigure[\label{fig:atomic:RAR}RAR atomic violation]{\includegraphics[width=0.25\textwidth]{figs/RAR.png}}&
      \hspace{-7mm}\subfigure[\label{fig:atomic:WAR}WAR atomic violation]{\includegraphics[width=0.25\textwidth]{figs/WAR.png}} \\
    \end{tabular}
  \end{center}
  \vspace{-0.5cm}
  \caption{\label{fig:atomic} Different types of atomic violation}
\end{figure}
As in Figure~\ref{fig:atomic:RAW}, in thread 1, a global pointer is set to point to aptr and then, dereference it and give the value to a local variable tmp. In thread 2, the global pointer is initialized to NULL. Consider the situation that the first sentence of thread 1 is executed and then thread 2 is executed. When ptr is dereferenced and give its value to tmp, segmentation fault happens. In this case, roll back thread 1 until thread 2 finishes and do the given value part. The bug can be fixed.


As in Figure~\ref{fig:atomic:WAW}, if thread 2 happens before the end of thread 1, error happens. In this case, rolling back thread 2 can fix the bug.


As in Figure~\ref{fig:atomic:RAR} and Figure~\ref{fig:atomic:WAR}, still rolling back one thread can fix the possible concurrency bugs. In general, for atomic violation bugs, rolling back one thread is sufficient to fix them.

\begin{figure}[t]
\centering
\includegraphics[width=0.5\textwidth]{figs/order_violation.png}
\caption{Order violation bug}
\label{order violation}
\end{figure}

\subsubsection{Recovering order-violation bugs}
Another kind of concurrency bug is called order-violation. That is one thread is required to be finished before another one. For example, in Figure~\ref{order violation}, A is required to finish before B. In this case, roll back B until A finishes can fix the bug.

\subsubsection{Recovering deadlock bugs}
A very common concurrency bug in multi-thread programming is deadlock problem. As in Figure~\ref{deadlock}, thread A holds lock 1, thread B holds lock 2 and thread C holds lock 3. A still wants lock 2, while it's held by B. So A is blocked. Similar situation happens in B and C. In this case, roll back any thread can fix deadlock. If roll back B, B releases lock 2, so that C can finish and release lock 3 and 2. Then B can finish. Finally, A can get all resource it wants and finish itself.
\begin{figure}[t]
\centering
\includegraphics[width=0.5\textwidth]{figs/deadlock.png}
\caption{Deadlock}
\label{deadlock}
\end{figure}

\subsection{Observation II}
\label{sbii}
For observation II, it indicates that roll back certain area of one thread is
sufficient to fix bugs and that area is defined as idempotent region.\\

\textbf{Idempotent region}:a code region that can be re-executed for any number of times without changing the program semantics.


It should end before the failing point because it is meaningless to roll back
the error part. Besides it, the idempotent region should not contain any writes
to shared variables. If it is, roll back it will continuously change a global
variable, which is unwanted by users. Furthermore, it will not contain any I/O
operations. According to investigation, only 15\% concurrency bugs contain I/0
operation. In order to guarantee the performance of ConAir, I/O operation is
eliminated from idempotent region. Also, it should not contain any writes to
local variables that could cause incorrect execution. This rule is also added to
guarantee the correctness of the ConAir. Some local variables, such as static
ones is initialized once and changed extends for the entire run of the whole
program. If it is included in the idempotent region, potentially incorrect
output may happen.

All rules are set to guarantee the correctness and performance of the usage of
ConAir. Above all, the working principle of the ConAir is to roll back a single
thread (failing thread) with its idempotent region.

\section{Design and Implementation}
\label{chp:Design}
According to the working principle of ConAir, there are three main challenges of the design of ConAir.
\begin{itemize}
\item
How to decide the failing point of the program
\item
How to find the idempotent region of the thread
\item
How to realize roll back step
\end{itemize}
\subsection{Identify Failure Sites}
ConAir discusses some types of common errors, including the assertion failures,
wrong output, segmentation fault and deadlock. Most concurrency bugs happen in
the former types. For the assertion failures, the tool will identify the
invocation of \_\_assert\_fail(), if it happens, it means assertion error
happens. For the wrong output error, it is very hard for the tool to detect it
because it does not know what the correct output should be. If user can add
assertion in the program to tell the tool what the correct output should be.
Then the tool can identify the failure site as mentioned. For the segmentation
faults, ConAir identifies every dereference of a heap/ global pointer variable
as a  potential segmentation fault failure site. For the deadlock, ConAir
changes every every pthread\_mutex\_lock into pthread\_mutex\_timelock. If it
timeouts, ConAir detects a deadlock problem in the thread.

\begin{figure*}[htbp]
\center
\begin{tabular}[c]{cccc}
\begin{minipage}{0.23\textwidth}
\begin{lstlisting}[language=C]
//assert(e)

  if(e){
  }else{
Failure:

  __assert_fail(...)

  }
\end{lstlisting}
\captionof{figure}{Assertion Failures}
\end{minipage}
&\begin{minipage}{0.23\textwidth}\begin{lstlisting}[language=C]
//printf("..",e,..);

  if(Assert(e)){
  }else{
Failure:
//developers specify
//Assert(...)
  }
  printf("..",e,..);

\end{lstlisting}
\captionof{figure}{Wrong Outputs}
\end{minipage}
&\begin{minipage}{0.23\textwidth}\begin{lstlisting}[language=C]
//tmp=*G_ptr;
  l_ptr=G_ptr;
  if(l_ptr>LowerBound){
  }else{
Failure:


  }
  temp=*l_ptr;

\end{lstlisting}
\captionof{figure}{Seg Faults}
\end{minipage}
&\begin{minipage}{0.23\textwidth}\begin{lstlisting}[language=C]
//pthread_mutex_lock(..);
  int ret = pthread_mutex_timedlock(..);
  if (ret!=TIMEOUT){
  }else{
Failure:


  }
\end{lstlisting}
\captionof{figure}{Deadlock Failures}
\end{minipage}
\end{tabular}
\end{figure*}
%\begin{figure*}[htbp]
%  \begin{center}
%    \begin{tabular}{cccc}
%      \subfigure[Assertion Failures]{\includegraphics[width=0.24\linewidth]{figs/assertion.png}}&
%      \hspace{-7mm} \subfigure[Wrong Output Failures]{\includegraphics[width=0.24\linewidth]{figs/wrong_output.png}} &
%      \hspace{-7mm} \subfigure[Segmentation Faults]{\includegraphics[width=0.24\linewidth]{figs/segmentation_fault.png}}&
%      \hspace{-7mm} \subfigure[Deadlock Failures]{\includegraphics[width=0.24\linewidth]{figs/deadlock_failure.png}} \\
%    \end{tabular}
%  \end{center}
%  \caption{Failure sites for different types of failures}
%  \vspace{-0.5cm}
%\end{figure*}

\subsection{Identify Idempotent Region}
It is very hard to defect failure sites in source code because it is written by users directly without a uniform type. So ConAir chooses to analyze the code in bitcode level. It uses LLVM to transform source code into bitcode. According the rules mentioned in Chapter ~\ref{sbii}, in bitcode level, there still exists rules:\\
\textbf{Idempotency-destroying} with LLVM bitcode:\\
(1)Writes to global or heap variables;
(2)Writes to local variables that are not allocated in virtual registers (static single assignment);
(3)Function-call instructions;
Eliminate codes which are against idempotency, idempotent region in bitcode level can be found.
\subsection{Realize Roll Back Step}
\begin{figure}
  \begin{center}
    \begin{tabular}{cc}
      \subfigure[\label{fig:roll:1}The original
      code]{\includegraphics[height=60mm,width=42mm]{figs/rollback_1.png}}&
      \hspace{-7mm}\subfigure[\label{fig:roll:2}The code with roll back]{\includegraphics[height=60mm,width=42mm]{figs/rollback_2.png}} \\
    \end{tabular}
  \end{center}
  \vspace{-0.5cm}
  \caption{\label{fig:roll}ConAir code transformation for assert}
\end{figure}
%\begin{figure}
%\begin{minipage}[t]{0.48\linewidth}
%\centering
%\includegraphics[width=\textwidth]{figs/rollback_1.png}
%\caption{The Origin Code}
%\label {OC}
%\end{minipage}%
%\begin{minipage}[t]{0.48\linewidth}
%\centering
%\includegraphics[width=\textwidth]{figs/rollback_2.png}
%\caption{The Code with Roll Back}
%\label{After OC}
%\end{minipage}
%\end{figure}
The roll back is realized by two functions in C++, named setjmp() and longjmp().
longjmp() is inserted in the line before failing site and setjmp() is inserted
in the beginning of idempotent region. When program goes to longjmp(), it will
jump to setjmp() instead of carrying on. Then the roll back is realized.Figure
\ref{fig:roll:2} is the code which ConAir execute and the Figure
\ref{fig:roll:1} is the origin one.


\section{Infrastructure}
In this chapter we will discuss our infrastructure setup for building and
running ConAir, and the toolchain of using the ConAir tool.

\subsection{Infrastructure Setup}
One of the standards that was mentioned previously to evaluate the tool is
Compatibility, which means the tool should have no OS/hardware modification.
When we try to build and run the ConAir tool, however, we find out that although
it does not require any OS/hardware modification, it has very strong requirement
to the OS and corresponding software installed.

We make many effort in order to make Conair compile and run. For example, We try
four different versions of Linux: Ubuntu 12.04 x64, Ubuntu 12.04 x86, the
Lonestar Cluster of UT, as well as CentOS 5.9. The first two are the OS that we
have as a virtual machine. None of them work perfectly. Then we ask the
author and they provide us the specific OS version that they use. Then we try
Lonestar Cluster which has an OS pretty similar to the one that the author
provides, but we do not have the previlidge to install the required runtime
library. Finally we tried CentOS 5.9 which is exactly the same as the author
suggests, but we still cannot fix the compatibility issue (missing GLIBCXX\_3.4.9
and GLIBC\_2.7).

We also try different versions of LLVM, llvm-gcc and gcc. Specifically, we try
two versions of LLVM, llvm-2.8 and llvm 2.9; we try three versions of llvm-gcc:
the binary of llvm-gcc-4.2-2.8, binary of llvm-gcc-4.2-2.9, and llvm-gcc-4.2-2.8
built from source code; also we try five different versions of gcc: 4.7.3,
4.6.3, 4.4.3, 4.2.1 and 4.3.1 from source. For each try we do our best to remove
any compilation or dependency issue, and we try many combination between
different OS, LLVM, llvm-gcc and gcc. This process is very tedious and it take
more than 50\% of our time for this project. However, although we try very hard
we still cannot find a combination that is perfectly working. The main problem
we face is that the front end of LLVM cannot treat floating point global
variables appropriately. We believe that is because the version of LLVM we use
is too old, but ConAir does not support newer version of LLVM.

Fortunately, although we have the above issue, for normal multithread programs
without floating points we are able to make ConAir compile and run. Therefore we
can still do some experiments to demonstrate the effect of ConAir. We will show
our experiment results in section~\ref{sec:experiment}

\subsection{Toolchain}
There are several steps needed to make ConAir work, and
Figure~\ref{fig:toolchain} shows the structure of the ConAir ToolChain. Here we
want to generate two executables, the original one and the fixed one, so that we
can compare and see if the concurrency bugs have been fixed or not.\\
\begin{figure}[htbp]
\centering
\includegraphics[width=0.5\textwidth]{figs/toolchain.png}
\caption{Toolchain of ConAir}
\label{fig:toolchain}
\end{figure}
\textbf{Generating the original executable}
In order to generate the original executable, we first use \textbf{llvm-gcc}, a
LLVM front end compiler, to compile the C source code \textit{Hello.c}, into
LLVM bitcode \textit{Hello.bc}. LLVM bitcode is a high level intermediate
representaton (IR) of LLVM. It will unify source code in different languages
into a common source code, with no optimization applied.

After the LLVM bitcode is generated, we then use \textbf{llvm-dis}, a LLVM
diassembler, to diassemble the above bitcode into LLVM assembly code
\textit{Hello.ll}. After that, we will use \textbf{llc}, a static compiler of
llvm, to generate the above assembly code into another assembly code for a
specified architecture \textit{Hello.s}. Finally, we will use another front end
compiler \textbf{llvm-g++} to compile the above assembly code into the original
executable \textit{Hello.init}\\

\textbf{Generating the fixed executable}
In order to generate the fixed executable, we reuse the same bitcode
\textit{Hello.bc} above. We use LLVM \textbf{opt}, which is the LLVM optimizer,
to load the ConAir library first. This step will perform static code analysis of
the above bitcode, identify the failure site and idempotent region, and do the
code transformation by inserting setjump and longump to the failure site and
idempotent regions. All above process is done in the bitcode level, and in the
end it will generate a new bitcode called \textit{Hello1.bc}.

After the new bitcode is generated, this time we directly use \textbf{llvm-g++}
together with the runtime libraries of ConAir loaded by the
\textbf{-use-gold-plugin} option, to generate the fixed executable
\textit{Hello.final}.

In later experiments for each program we will generate both the original and
fixed executables, and compare their performance.


\section{Experiment}
\label{sec:experiment}
With all infrastructure and toolchain prepared, we perform a series of experiment
to demonstrate that ConAir is able to fix concurrency bugs. We will first
illustrate the experiment results from the author of ConAir. After that we will
show our own experiment results for non-deadlock, deadlock programs and programs
with other concurrency bugs.

\subsection{Original result}
Table~\ref{tab:original} lists the experiment result from the author. In the
ConAir paper the author evaluated ConAir on 10 real life programs, whose lines
of code varies from 1.2K to 693K. All these programs have known failures to
ConAir. The author run ConAir on these programs in two modes. The fix mode is
used when the programmer know where the bug is, and manually insert some
assertions to help ConAir identify and fix the bug; The survival mode is to run
ConAir without any information of the program, and let ConAir find and fix all
potential bugs on its own.

% Table generated by Excel2LaTeX from sheet 'Sheet1'
\begin{table}[htbp]
  \centering
    \begin{tabular}{|r|r|r|r|r|} \hline
    App. & LOC & Failures & Fix & Survival \\ \hline
    FFT & 1.2K & w.output & Yes & Conditional \\
    HawlNL & 10K & Hang & Yes & Yes \\
    HTTrack & 55K & Seg.fault & Yes & Yes \\
    MozillaXP & 112K & Seg.fault & Yes & Yes \\
    MozillaJS & 120K & Hang & Yes & Yes \\
    MYSQL1 & 682K & w.output & Yes & Conditional \\
    MYSQL2 & 693K & assertion & Yes & Yes \\
    SQLite & 95K & assertion & Yes & Yes \\
    Transmission & 67K & hang & Yes & Yes \\
    ZSNES & 37K & assertion & Yes & Yes \\ \hline
    \end{tabular}%
  \caption{Experiment result from the author}
  \label{tab:original}%
\end{table}%


From Table~\ref{tab:original} we can observe that ConAir is able to fix all
concurrency bugs for all these 10 real life programs in fix mode. However, for
wrong output failure, ConAir can only conditionally fix the bug in survival
mode, which means that ConAir needs additional assertion information to know what
the correct output should be, otherwise ConAir cannot fix the bug. This
experiment shows that ConAir is capable of recover and fix concurrency bugs of
all know types, regardless of program sizes.

\subsection{Non-deadlock programs}
In order to further demonstrate how ConAir works for fixing different bugs, we
come up with our own concurrency programs with different types of bugs, and run
ConAir to see if it can fix the bug or not. Table~\ref{tab:nondl} lists all
programs with their corresponding bugs, and the result of running the original
executable and fixed executable. Since concurrency bugs do not always happen,
for each program we let the program run for 1000 types in fix mode, which means
  we add some assertions for the wrong output bug, and we record the number of
  correct outputs.

% Table generated by Excel2LaTeX from sheet 'Sheet2'
\begin{table}[htbp]
  \centering
    \begin{tabular}{|r|r|r|r|} \hline
    Program & Failures & Before fix & After fix \\ \hline
    WAW1 & w.output & 31/1000 & 1000/1000 \\
    WAW2 & w.output & 983/1000 & 1000/1000 \\
    RAW1 & Seg fault & 938/1000 & 1000/1000 \\
    RAW2 & Seg fault & 998/1000 & 1000/1000 \\
    RAR1 & Seg fault & 1000/1000 & 1000/1000 \\
    RAR2 & Seg fault & 1000/1000 & 1000/1000 \\
    WAR1 & assertion & 16/1000 & 1000/1000 \\
    WAR2 & assertion & 976/1000 & 1000/1000 \\ \hline
    \end{tabular}%
  \caption{Experiment result for non-deadlock programs for 1000 runs each}
  \label{tab:nondl}%
\end{table}%


From Table~\ref{tab:nondl} we can see that ConAir is capable to fix all three
known type of non-deadlock concurrency bugs. For example, for WAR1 only 31 out
of 1000 runs give correct output for the original executable. However, all 1000
runs give correct output after fix. We can observe similar behaviors for other
programs and other bugs. One thing to note is for RAR1 and RAR2 even for the
original program we cannot observe any error in 1000 runs, although we are sure
theoretically segmentation faults will happen, therefore it is not a counter
example, therefore it is not a counter example.

\subsection{Deadlock programs}
Different from non-deadlock programs, deadlock programs will usually hang up
instead of halt immediately and give the wrong output. Therefore we design
another experiment to demonstrate the ability of ConAir to fix deadlock bugs.
We come up with two simple programs with deadlock. A code snippet of Deadlock1
is shown in Figure~\ref{fig:deadlock-code}.
\begin{figure}[htbp]
\begin{center}
\begin{lstlisting}[language=C]
              void thread1(void){
                pthread_mutex_lock(&lock1);
                pthread_mutex_lock(&lock2);
                pthread_mutex_unlock(&lock1);
                pthread_mutex_unlock(&lock2);
                printf("t1 done\n");
              }
              void thread2(void){
                pthread_mutex_lock(&lock2);
                pthread_mutex_lock(&lock1);
                pthread_mutex_unlock(&lock1);
                pthread_mutex_unlock(&lock2);
                printf("t2 done\n");
              }
\end{lstlisting}
\caption{Code snippet of a deadlock bug}
\end{center}
\label{fig:deadlock-code}
\end{figure}

We run ConAir on the two programs with deadlock for 1000 times, and we will
assume no deadlock happens if it does not hang for all 1000 times. The results
are shown in Table~\ref{tab:dl}.

% Table generated by Excel2LaTeX from sheet 'Sheet3'
\begin{table}[htbp]
  \centering
    \begin{tabular}{|r|r|r|} \hline
    Program & Before fix & After fix \\ \hline
    Deadlock1 & Hang & Hang \\ \hline
    Deadlock2 & Hang & Hang \\ \hline
    \end{tabular}%
  \caption{Experiment result for deadlock programs}
  \label{tab:dl}%
\end{table}%


From Table~\ref{tab:dl} we can see that both the initial executable and fixed
executable has deadlock programs. We tried to look into the ConAir code as well
as its configuration file, and we believe all of our settings are correct, and
therefore it might be that ConAir cannot fix the deadlock program that we have.

We contacted the author with this question, together with our command,
configuration file and our buggy programs. They told us that for running ConAir
for deadlock programs some other commands need to be added, and they will
  provide me some further explanation later. Therefore, we still believe that
  ConAir is capable to fix some deadlock bugs, especially the one that we have. However, in the scope of this project we cannot demonstrate it yet.

Also, in the ConAir paper the author claims that ConAir is not able to fix all
deadlock bugs, if the deadlock cannot be fixed by re-execute the idempotent
region. This claim make us to think that there may be other concurrency bugs
that ConAir cannot fix. We design another set of experiment in
Section~\ref{sec:ex-other}, where we show how ConAir performs for other unknown
types of concurrency bugs.

\subsection{Programs with other concurrency bugs}
\label{sec:ex-other}
% Table generated by Excel2LaTeX from sheet 'Sheet4'
\begin{table}[htbp]
  \centering
  \caption{Add caption}
    \begin{tabular}{rrrr}
    \toprule
    Program & Failures & Before fix & After fix \\
    \midrule
    bug1 & More than one thread is waiting for a Cond signal, but only one is sent & Hang & Hang \\
    bug2 & It has exhausted the default thread stack space & Seg.fault & Seg.fault \\
    bug3 & Shows an unsafe way to pass arguments in the pthread\_create, pass a memory location which value is not unique & w.output & Try to fix but failed \\
    bug4 & Synchronization problem,  signal thread calling before waiting & Hang & Fix \\
    bug5 & Missing a pthread\_exit call at the end of main & w.output & w.output \\
    bug6 & Read and write critical section without synchronization & w.output (random) & w.output (stable) \\
    \bottomrule
    \end{tabular}%
  \label{tab:addlabel}%
\end{table}%

In the above experiments we show that ConAir is able to fix all known
non-deadlock bugs, and we still believe it can fix the deadlock bugs if we use
the right command. However, we would also want to understand how ConAir deals
with other types of concurrency bugs.

However, it is not easy to find a good benchmark that satisfies our requirement.
Most of the state-of-art concurrency benchmarks are designed to test the
performance of CPU simulators, and they are all bug free. We do some literaturral
research but cannot find any simple set of buggy multithreaded programs.
Fortunately, we are able to find a pthread programming
exercise~\cite{pthread-web}, where it provides six buggy multithreaded programs
with very different types of bugs from the ones that are supported by ConAir.
Table~\ref{tab:other} lists all of the six programs and the description of each
corresponding bug, as well as the execution result before and after running
ConAir on the buggy program.

From Table~\ref{tab:other} we can see that ConAir cannot fix most of the
unknown bugs. For bug1, bug2 and bug5 the results before and after fix are
exactly the same. However, we could observe that although some bug types are
unknown, ConAir will try to fix them. For example, for bug3 we can observe from
the result that ConAir is trying to insert the setjmp() and longjmp() function,
but just cannot fix that. For bug6 ConAir is providing a more stable output,
compared with a pure random output before fix. For bug 4 ConAir can even fix the
bug without knowing it can supporting that type of bug.

\section{Discussions and Future Work}
In this project we explored the state-of-art research tool ConAir, which is used
for concurrency bug recovery. We believe that there are many dimensions that the
  current work for both ConAir and our project can be extended.

\subsection{Better compatibility}
In this project we have a very painful experience setting up ConAir and make it
running. One of the reasons is that ConAir is built based on old version of
Linux and LLVM, and it does not recognize the latest compilers and libraries.
Since ConAir is a very effective tool, it is worthy to spend time to maintain
this tool so that it is compatible for modern OS/software, and therefore it
could be more widely used.

\subsection{New type of concurrency bugs}
From our experiment result in Section\ref{sec:ex-other} we can see that for some
other types of concurrency bugs ConAir is still capable in fixing some of them.
It is promising that ConAir being extended to support new types of concurrency
bugs. For example, bug4 in previous experiment has a synchronization problem,
which is not any of the four known concurrency bugs. However, ConAir is able to
fix that bug. In the future we can identify more such bugs and extend ConAir to
add support to those bugs.

\subsection{Concurrent program generator}
In this project we generate all our buggy concurrent programs by hand. However,
during this process we feel that many of these programs are very similar in
format. It motivates us to have an idea of building a program generator for
concurrent programs for both bug-free and buggy programs. Specifically, we could
provide some parameters, such as number of threads, the work to be performed for
each thread, different thread scheduling, etc. If this current program generator
can be built, it could be used to evaluate many concurrent program testing and
bug recovery tool such as ConAir, and it will be of grate value for many
researches.

\section{Conclusion}
In this project we have a chance to explore the state-of-art research tool
ConAir, which is a tool used for concurrency bug recovery. We read through the
ConAir paper and have a good understanding of the behind working principles of
it, such as single thread rolling back and idempotent region. From the
ConAir paper we know that it is capable of recovering concurrent bugs of many
real-life programs, and in the experiment we demonstrate that it is able to fix
concurrency bug with assertion failure, segmentation fault and wrong output.
Although we cannot demonstrate ConAir's ability to fix deadlock bugs, we still
believe that is due to the settings that we use rather than the ConAir itself.
Also, we find out that other than the four types of bugs specified in the ConAir
paper, it is possible for ConAir to fix, or at least try to fix, some other
unknown types of concurrency bugs. We think ConAir is a very powerful tool, and
it would be applied to many real-life programs if it can be made compatible with
the latest OS and compiler.

Other than having a chance to use ConAir, we also enhanced our understanding of
pthread programming from this project by understanding ConAir and writing our
own multithread programs. Also, by setting up and compiling ConAir we enhanced
our understanding of Linux and LLVM infrastructure.

\section{Acknowledgements}
We would like to thank Dr. Wei Zhang and Prof. Shan Lu from Computer Science
Department of University of Wisconsin-Madison for providing us the source code
of ConAir. Also, they offered us lots of help when we have problem setting up and
running ConAir.


\bibliographystyle{unsrt}
\bibliography{Reference}

\end{document}
