\chapter{Conclusion}
\label{chp:conc} We have presented OSS as a new computable platform
for the uniform and consistent measurement of concept similarity and
combination. The platform, aiming at a high coverage of semantic
space with a minimal concept set, shapes the ways of modeling
concept inter-relatedness, while providing guideline for concept
development. To show the feasibility of OSS, we explore and
experiment search related tasks including word sense disambiguation
and concept selection. Our findings show that, due to the uniform
way of assessing similarity, OSS is a feasible solution for
large-scale video search and concept combination. Currently we
assume that OSS exists in a linear space for computational reason.
Whether a nonlinear space assumption is feasible for OSS remains an
unanswered issue that worths further investigation.

A useful resource currently not explored in OSS is the co-occurrence
statistics of concepts in video data. The statistics can be directly
utilized for basis concept selection, amending the semantic space
such that the co-occurred behavior can also be modeled. Under such
circumstance, the space is enriched with both ontology semantic and
statistics useful for video search. Developing the basis concepts in
this space as detectors could be more realistic since the statistics
indeed hint the utility and observability of the concepts.
%
In addition to positively correlated concepts, the set of negative
concepts (e.g., {\em indoor} versus {\em outdoor}) is also a useful
piece of information for fast pruning in video search as presented
in \cite{W.H.Lin:ICME:2006}. It is possible to have another
``negatively correlated'' semantic space, complementary to OSS, to
allow fast filtering one on hand, and effective searching on the
other hand. We will consider both aspects (co-occurrence and
negative correlation) as the future extension of OSS.
%
