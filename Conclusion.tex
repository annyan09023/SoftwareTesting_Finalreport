\section{Conclusion}
In this project we have a chance to explore the state-of-art research tool
ConAir, which is a tool used for concurrency bug recovery. We read through the
ConAir paper and have a good understanding of the behind working principles of
it, such as single thread rolling back and idempotent region. From the
ConAir paper we know that it is capable of recovering concurrent bugs of many
real-life programs, and in the experiment we demonstrate that it is able to fix
concurrency bug with assertion failure, segmentation fault and wrong output.
Although we cannot demonstrate ConAir's ability to fix deadlock bugs, we still
believe that is due to the settings that we use rather than the ConAir itself.
Also, we find out that other than the four types of bugs specified in the ConAir
paper, it is possible for ConAir to fix, or at least try to fix, some other
unknown types of concurrency bugs. We think ConAir is a very powerful tool, and
it would be applied to many real-life programs if it can be made compatible with
the latest OS and compiler.

Other than having a chance to use ConAir, we also enhanced our understanding of
pthread programming from this project by understanding ConAir and writing our
own multithread programs. Also, by setting up and compiling ConAir we enhanced
our understanding of Linux and LLVM infrastructure.

\section{Acknowledgements}
We would like to thank Dr. Wei Zhang and Prof. Shan Lu from Computer Science
Department of University of Wisconsin-Madison for providing us the source code
of ConAir. Also, they offered us lots of help when we have problem setting up and
running ConAir.
