\prefacesection{Abstract}

Multimedia-based ontology construction and reasoning have recently
been recognized as two important issues in video search,
particularly for bridging semantic gap. The lack of coincidence
between low-level features and user expectation makes concept-based
ontology reasoning an attractive mid-level framework for
interpreting high-level semantics. In this report, we propose a
novel model, namely ontology-enriched semantic space (OSS), to
provide a computable platform for modeling and reasoning concepts in
a linear space. OSS enlightens the possibility of answering
conceptual questions such as a high coverage of semantic space with
minimal set of concepts, and the set of concepts to be developed for
video search. More importantly, the query-to-concept mapping can be
more reasonably conducted by guaranteeing the uniform and consistent
comparison of concept scores for video search. We explore OSS for
several tasks including concept-based video search, word sense
disambiguation and detecor fusion. Our empirical findings show that
OSS is a feasible solution to timely issues such as the measurement
of concept combination and query-concept dependent fusion.
